\documentclass[class=article, crop=false, draft=true]{standalone}
\usepackage{graphicx}
\graphicspath{{../Figures}}

\usepackage{cleveref}

\begin{document}
\label{chap:cases}
This chapter will go through the two scenarios to be simulated, their setups and purposes. The scenarios are divided into several cases so that each case only refers to one simulation setup.

\section{USV mastered control response in different seastates}


\section{Vertical response in rough seas}
As the surface vessel moves up and down with the waves, the ROV will be tugged by the cable. This would potentially change its depth. However, the cable is long and elastic (to an extent). This scenario will use the maximum acceptable wave range from the previous scenario and simulate the ROV at three different depths. 10m, 50m and 100m. These depths have been chosen because the current planned max depth for the ROV is 100m, the other points being chosen as respectively a point reasonably close to the surface and the midpoint between the maximum extent and the surface.

What this scenario might show is what sort of considerations might have to be made with regards to the control requirements for the crane and the ROV. Initially, the plan is that the crane would follow low-frequent disturbances while the ROV would follow high-frequency but lower amplitude disturbances. The efficacy of this system can be tested here.

Furthermore, these scenarios might be used to decide what elasticity is acceptable for the cable, as well as further allowing choosing operational criteria for the mission. The acceptable wave disturbance is likely higher for transit than it is during ROV operations, this scenario will be able to further illucidate this.


\end{document}
