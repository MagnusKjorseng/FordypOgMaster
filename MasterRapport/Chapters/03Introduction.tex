\documentclass[class=article, crop=false]{standalone}
\usepackage{graphicx}
\graphicspath{{../Figures}}

%\usepackage[round]{natbib}   % omit 'round' option if you prefer square brackets
%\bibliographystyle{plainnat}

\begin{document}
This thesis concerns the development and testing of a control system for a sub-surface marine plastic waste removal system. The waste removal system is a part of the project under the name "Plan Sea", and due to the deep ties between the project and this thesis and its related project, Plan Sea will be mentioned frequently and explained intermittently.

This chapter goes into the problem of marine plastic pollution, how to avoid and remove marine plastic waste, as well as going further into the Plan Sea project and defining the exact specifications for the project this thesis is based on.


\section{Marine plastic pollution}
Marine plastic pollution is a widely documented issue. The exact amount of plastic is not known. The \citet{world_economic_forum_top_2022} estimate there to be betweel 75 and 199 million metric tons of plastic waste currently in the ocean, while \citet{jambeck_plastic_2015} estimated that the annual release in 2010 alone was between 4.8 and 12.7 metric tons. According to \citet{isobe_fate_2022}, most plastic sinks, however the majority of research on marine plastics is done on surface plastics. Subsurface plastics will leech and decay into the surrounding ocean. This is bad. Removing the plastics will stop them from leeching and decaying into the sea.


\section{Why's it an issue}
Health effects, effects on animals etc., effects on the conditions in the ocean (acidity etc.). There's also very little research going into it currently.

\section{A layered approach}
Solving the problem of marine plastic pollution can be seen as a layered approach. In the first layer is reducing the use of plastics in general. Particularly single use plastics should be reduced as much as possible, but more sustainably produced options made from natural materials should also be considered for things like clothing, crates, liners for fluid tanks or other places where plastics are used today.

Still, there are some areas of life in which the use of plastics or single-use plastics cannot be removed practically. Healthcare is an example of this, where single-use items are necessary for hygeine considerations, as well as for practical purposes. For instance having intravenous needles not made from plastic was done before, but they are far less comfortable than their modern plastic counterparts.

In cases like these, as well as others, the second layer steps in. The second layer here is increased or improved collection. Waste which is collected correctly doesn't end up in the natural environment. Having proper recycling efforts, as well as waste collection systems for consumers and industry is a massive boon here. Additionally under this layer is solutions such as river collection nets and similar. These solutions are made to catch waste which happens to get into the waterways or wastewater treatment system before it is released into the sea. However, environmental studies need to be performed to ensure undue ecological damage isn't done by essentially demming up rivers or waterways. There are also issues when it comes to the cost of maintenance and emptying of the collection systems.

\begin{figure}
	\centering
	\includegraphics{waste_net}
	\caption{Illustration image of a drainage net installed in the city of Kwinana, Australia. The nets installed are used to collect waste before it enters a natural reserve. Image credit of the City of Kwinana}
	\label{fig:waste_net}
\end{figure}

The two layers described above are helpful to prevent waste from getting into the oceans, as the saying goes "an ounce of prevetntion is worth a pound of cure". There are two problems with relying only on those two layers, however. Firstly, some litter inevitably slips through the cracks of the systems and ends up in the ocean despite our best efforts. This is unavoidable and will always be a problem so long as plastics exist. Secondly, and arguably more importantly: there are already many million tons of waste in the ocean. Even if our prevention efforts were perfectly implemented with no "leaks", the harm is already done, or in the process of being done as is the case with leeching. Thus, the third layer is necessary.

The third layer of this solution is removal. By removing the waste from the oceans, it is possible to halt its harmful future effects on the environment. Today, most collection efforts are done by human divers in relatively shallow waters, such as harbours. Human divers are an excellet solution to this problem when it comes to their dexterity and detection capabilities, however diving is a dangerous activity. Nitrogen saturation in joints and tissues is a very real danger for divers who spend too long pressurized or don't decompress properly. Diving in areas such as harbours is dangerous with regard to boat traffic potentially harming the divers. Generally the diving done for litter collection would also be done on a volunteer basis by divers with only recreational certification and equipment. This means that for practical purposes, they would be limited to a maximum of 50m depth. This is due to how air is used more quickly at greater depths, as well as the problems of decompression or tissue saturation already mentioned. Nitrogen narcosis, a condition where a diver experiences sudden drowsiness, inattentiveness or sluggishness is also a greater risk at greater depths. There exist solutions for diving deeper than 50m of course, but generally then one would use expensive gas mixes rather than plain or oxygen-enriched air, as well as spending more time on descent and ascent for safety reasons. Saturation diving is another option for diving deeper, but it is so insurmountably expensive for litter collection that it is not a real option. Saturation diving generally requires specialized ships to stay in position for weeks or months at a time with on-board pressure chambers and diving bells for crews to go down. The wage costs alone would be prohibitive for such a non-profitable effort as litter collection.

Clearly, there is a need for a solution which works at greater depths than humans can reach, and which is relatively cheap to implement and operate. This is where the Plan Sea project comes in. Plan Sea is a student-driven project which is undertaken at NTNU in Ålesund, Norway, and is the basis for this thesis. The Plan Sea project will be further discussed later. The third layer is also going to be the main focus of this thesis going forward, further options for reduction or prevention (layers 1 and 2) will not be discussed at length.

Further in this thesis, plastic litter will be divided into four main categories, differentiated by size along the largest axis. The categories and their deciding size can be seen in \cref{tab:plastics}. This is a common distinction used in several articles discussing marine plastic pollution.

\begin{table}
\centering
\begin{tabular}{c|c}
\textbf{Plastic category} & \textbf{Size criteria} \\
\hline
Micro & \(l<5mm\) \\
Meso & \(5mm<l<50mm\) \\
Macro & \(50mm<l<500mm\)\\
Mega & \(500mm<l\)\\
\end{tabular}
\caption{The four categories of plastic sizes used in this thesis, \(l\) representing the length along the longest axis of the litter}
\label{tab:plastics}
\end{table}

\subsection{Potential removal strategies}
The different sizes and types of litter necessitate a different approach for removing them. A tyre, used as a boat fender which fell off a boat, does not have the same removal strategy as a discarded fishing net or a bed of settled plastic dust. While large, discrete litter conceptually can be removed fairly simply, by using a large gripper for example. Complex or long shapes, like discarded nets, fishing lines or mooring lines, need different considerations because of a risk of entanglement with the removal equipment. Long shapes also need to be further considered with regards to how they have integrated into the seafloor already, if removal is possible without disturbing the sealife that might have taken home in it, or if it might need to be cut into several parts for material safety. Microplastics again require a new strategy. It's possible to trawl with very fine nets, though they would likely fill up very quickly due to the necessarily fine mask of the net to catch the plastics. An alternate option is a vacuum solution with a sort of size filtering option, a cyclonic filter for instance.

The end result is that collecting meso- and microplastics, as well as large aspect ratio litter can be dangerous to equipment or require a lot of specialized development only for the collection equipment. It was decided that for the Plan Sea project, macro- and megaplastics were going to be the main focus of removal because these can be removed with a simple gripper/grasper/bucket solution.

\subsection{Plan Sea project}
\begin{figure}
	\centering
	\includegraphics[width=0.6\textwidth]{overview}
	\caption{Sketch of the proposed solution from the Plan Sea project. The sketch includes people on the surface vessel, these are only meant for scale as the planned vessel will be unmanned.}
	\label{fig:overview-sketch}
\end{figure}
The Plan Sea project is, at time of writing, an ongoing student-led project at NTNU in Ålesund. The goal of the project is to find a solution to the problems of subsurface marine plastic pollution as laid out above. In the project, several specialized groups have been working on separate fields. There has been a group dedicated to hull design, a group dedicated to energy- and propulsion systems, as well as groups for the mechatronic systems, the automation systems and systems for water-quality testing and collection.

During the first year of the project, the result of the Plan Sea project has been designs for a solution, as well as primary material acquisition. The goal is to build an autonomous catamaran surface vessel which will work as a stable platform for a non-buoyant ROV which will descend through the water column and detect and collect macroplastics. In addition, a subsea basket solution was proposed which would reduce the amount of transit time for the ROV. A sketch of the total Plan Sea solution can be seen in \cref{fig:overview-sketch}. Generally, descent and ascent in marine ROV operations is measured on the order of meters per minute, this means that in order to descend to for instance 50m depth would generally take several minutes. If the goal of collection is for very large objects, for instance discarded rubber tyres or large fish crates, this might be worthwhile. However, to increase flexibility, the basket would make it worthwhile to collect smaller pieces as well, by having a place to return to underwater lets the ROV collect a piece of litter and place it in the basket, only ascending once the operation requires it.


At time of writing, the hull for the Plan Sea project has been completed, as well as an azimuth thruster solution. The vessel is built from donated carbon fiber sandwich boards, donated by Brødrene Aa shipyard. The thrust configuration is with two electric azimuth thrusters placed at the stern. The thrusters themselves are made by Torqeedo, and the azimuth solution is developed in-house at NTNU.

\section{Previous work (Specialization project)}

This report is in a way a continuation of a specialization project\cite{specialization} done previously. The specialization project has always had the goal of being the groundwork for this master's thesis. The goal of the specialization project was to set up a framework to make work for the master's project easier. Part of the specialization project has been starting work on a simulator that can be used to work on a control system. The control system is currently rudimentary implemented and there is room for improvement.

The goal of the development for the conttrol sytem is to have it be platform agnostic. That is, the control system does not care whether it's connected to a simulator or a vessel in the real world. The way this is ideally implemented is by using ROS2. ROS2 is a framework to control robotics, built around publishers and subscribers that all act around topics.

An ideal solution will have a

A major point of interest in the specialization project, along with making the groundwork for this project, was to find whether or not the towed ROV could be localized using the tether utilizing a sort of taut line localization \cite{app:taut-line}. This was found to be not possible due to the large delay in movement of the tether. As the ROV moves relative to the surface vessel or the other way around, the tether is affected by water resistance, leading essentially to a lag of the effects. If the ROV moves while the surface vessel stays stationary the tether follows the ROV and the ROV experiences this as an added resistance, but the surface vessel doesn't feel this effect until the effect has propogated all the way up the tether, up to several minutes later depending on the depth. The end result of this is that positioning and following for the surface vessel is essential for the ROV to be able to do its work. The surface vessel should follow the ROV's movement fairly closely, and the acceptable range of movement for the ROV before the surface vessel would need to follow should be essentially a cone shape with the tip at the surface vessel.

\section{The focus of this project}
This thesis is focused on a project related to the Plan Sea project. Plan Sea in itself is too large to do as a single-person-project, and thus this is only part of it. The goal for the project this thesis describes is to develop a control system which is extensible and usable for the full system, including surface vessel, ROV and crane. Additionally, the project has had as a goal to continue the development of the simulator from the specialization project to the point where the simulator and the real-life solution are equivalent from the point of view of the control system. This would allow for simpler development of systems and tuning of the controls by providing a realistic means of testing the development.

\section{Reader's guide}

%\bibliography{../bibliography}
\end{document}









\section{Marine plastic pollution}
Plastic pollution in the oceans has been widely documented, however the amount of plastic currently in the ocean is uncertain. Jambeck et al.\cite{jambeck_plastic_2015} estimates that in 2010, somewhere between 4.8 and 12.7 million metric tons(MT) of plastic ended up in the ocean. According to the World Economic Forum\cite{world_economic_forum_top_2022}, there is between 75 and 199 million MT of plastic waste currently in the ocean. Around two thirds of all plastics that end up in the ocean are heavier than seawater \cite{isobe_fate_2022} meaning that they sink and either drift in the pelagic zone or end up on the seafloor as litter. Removing litter and plastic pollution on a large scale is difficult, removing it under many meters of ocean makes it much more difficult. 

It is undesirable to have plastic waste in the oceans. This is because of the health effects the plastics have on marine and terrestrial life. Two points are especially of note: microplastics and leeching. Microplastics are plastic particles smaller than 5mm. Leeching on the other hand, is the plastics' chemical interaction with the seawater surrounding them, leeching harmful chemicals into the water.\cite{prata_environmental_2020}\cite{zolotova_harmful_2022}\cite{segovia-mendoza_how_2020}\cite{obuzor_chemical_2023} For both of these issues, the best solution is to remove the litter. This is because plastics' general longevity. For example, Oluwoye et al. \cite{oluwoye_degradation_2023} found that polyethylene, commonly used as a coating for subsea structures, would take about 800 years to degrade on the ocean floor. Polyethylene is also used in many consumer- and industrially facing applications, for instance in plastic cannisters for liquids, boxes and crates for fishing or other industrial practices, or as plastic bags. 

\section{The Plan Sea Project}
The desire to deal with sub-surface marine plastic waste, i.e. litter both in the pelagic zone and on the seabed, was what sparked the Plan Sea project. Plan Sea is a student driven project at NTNU in Ålesund with the goal of finding, developing and testing a potential solution for removing sub-surface plastics. The project is at time of writing still in its early phases and ongoing. At time of writing, a hull has been constructed from carbon fiber sandwich boards. Thrusters have been mounted to the hull and work on controlling them has started. Additionally, an ROV has been acquired for the project. 

Since this is not a report focused on the Plan Sea project, the proposed solution arrived at in Plan Sea will not be discussed in detail. However, because of the relationship between this project and Plan Sea, it is necessary to describe the solution at a surface level. 

\subsection{The proposed solution}
The solution which the Plan Sea project is aiming for is an ROV-based solution with an unmanned tender-vessel on the surface. The ROV has a gripper attached and will navigate to find litter, grab it and pick it up. The surface vessel exists to provide the ROV with a greater lifting capacity. If the ROV was to lift purely under the force of its own thrusters, as is traditional for ROVs, the total amount of lifting force available would be limited by the vertical thrust available. This would mean that either the ROV would have to have a very large amount of vertical thrust available relative to its size, or that the total lifting capacity would be very small, neither of which are desirable. By connecting the ROV to the surface vessel with a winch and a lifting cable, it is possible to use the ROV to do fine-navigation to find and attach to litter, and then use the lifting force of a winch and the total buoyancy of the surface vessel for lifting heavier objects. Ideally an ROV which originally only has a total lifting force of about 15kg would be able to lift heavy cables or car tires. Additionally, the solution is supposed to have an undersea basket for collection to avoid having to lower the ROV to the seabed and then lift it up to the surface for each piece of waste. Instead the ROV can be lowered down and can pick litter it finds and place it in the basket. Once the basket is full it can then be lifted up and either emptied on deck, exchanged for another basket or taken back to shore for further sorting there. A sketch of the solution can be seen in \cref{fig:overview}.

\begin{figure}
	\centering
	\includegraphics[width=0.5\textwidth]{overview}
	\caption{A sketch of the proposed solution for Plan Sea showing a surface vessel, a tethered ROV and a collection basket}
	\label{fig:overview}
\end{figure}

Using this solution allows for completely ignoring the buoyancy of the ROV, unlike traditional ROVs. Traditional ROVs are generally designed to be neutrally buoyant, meaning that they neither sink nor float, but keep their vertical position in water once placed there. Since the Plan Sea ROV will be attached to a cable to the surface vessel at all times, it can instead hang from the cable. This means that it's possible to attach larger grippers, more battery capacity, more detection/lighting/navigation equipment, and otherwise allows for any desired modifications to be done to the ROV. Additionally, since the ROV doesn't need to provide vertical thrust, it is much easier to not disturb the seabed which will provide a clearer view for detection equipment based on visible or near-visible light. However, having a non-buoyant ROV does come with some drawbacks.

One drawback of this solution is that it will switch between two operating modes, searching/grabbing and lifting. In the search/grab mode the ROV will be near-neutrally buoyant, or somewhat negative. When lifting the ROV might be severely negatively buoyant. These two wildly different operating modes increase the needed complexity of the system. Another drawback is that as the ROV is hanging from the cable, it creates a coupled system consisting of the surface vessel and the ROV, and necessitates the two moving together as one unit. The forces the surface vessel experiences, such as waves or wind, will impact the ROV, likewise currents or snags the ROV experiences will affect the surface vessel.

\section{Control systems}
A control system commands and regulates the behaviour of other systems automatically. For this project in particular, the control system will be in charge of maintaining and changing positions of the two vessels. A simplified function block diagram of the total system can be seen in \cref{fig:fbd}. The goal of the simulator is to function as a drop-in replacement for the vessel, local controllers and environmental impact shown in the figure. This makes it so that the development can happen digitally to then be quickly deployed in the real-world.

\begin{figure}
	\centering
	\includegraphics[width=0.8\textwidth]{control-fbd}
	\caption{A simplified function block diagram of the total system. Grey blocks are not currently implemented, green blocks are simulated and blue is for the controller which is system-agnostic}
	\label{fig:fbd}
\end{figure}


\subsection{Considerations because of a coupled system}
In the marine sector, dynamic positioning (DP) is commonly used. DP allows for a vessel to maintain a position or a course automatically despite external effects. This is used for example for offshore supply vessels which need to stay stationary relative to an anchored platform to allow loading and offloading of supplies. DP is also used for applications such as laying subsea fiberoptic cables, where maintaining correct speed and course is important to avoid damaging the cables. For the Plan Sea project too, a DP system is necessary because it consists of vessels that need to maintain specific positions at sea with wind, wave and current forces affecting the vessels

Normally a DP system only considers one vessel, however for the Plan Sea project it has to be more comprehensive than that because the two vessels are coupled. This is all further discussed in \cref{sec:math}

\subsection{The need for rapid prototyping}
Rapid prototyping is becoming increasingly popular with time. The goal of rapid prototyping is to create some simulated environment in which you can test and iterate on a solution until it is acceptable. Then, once you have a solution that works within a certain level of acceptability you can start to put materials and resources into building and implementing the solution in the real world. 

For my purposes, rapid prototyping will allow me to experiment with control system tuning and variables without having to deploy the full-scale vessel every time. Ideally, the solution arrived at in the prototyping stage will be directly applicable to the full-scale version, which allows for rapid deployment. The hope is that any issues will be detected and solved while testing digitally, meaning that we hopefully avoid large surprises during deployment. 

\section{Problem description}
A simplified sketch from \cref{fig:overview} can be seen in \cref{fig:simple}. It shows the three main components of this system: The surface vessel, the ROV, and the tether between them. It also shows the forces in the tether. As the tether holds the ROV up, the tether pulls the surface vessel down. This follows from Newton's second law of motion. The figure also shows the coupled nature of the system, since the tether will be taut at all times, both vessels will experience this force from the other at all times.

\begin{figure}
	\centering
	\includegraphics[width=0.6\textwidth]{simplified-overview}
	\caption{A simplified sketch of the problem, forces experienced by each vessel are shown in red. Forces are not to scale.}
	\label{fig:simple}
\end{figure}

In \cref{fig:angled} two scenarios are shown overlaid, one where the ROV is hanging straight below the surface vessel and another where it is at an angle. The result is that the ROV is not at a constant height. If we imagine a desired elevation above the seafloor is constant, then either the ROV needs to have more tether payed out and provide lift through its own thrusters, or the surface vessel needs to move to allow the ROV to hang perpendicular to the surface of the sea. If more tether was to be payed out and the system then stabilizes, the ROV will fall to the lowest possible point. It is possible then that the ROV might collide with the seabed or other objects. The ideal solution then becomes that the surface vessel follows the ROV, or the ROV only operates within a given area of operations directly underneath of the surface vessel. 

\begin{figure}
	\centering
	\includegraphics[width=0.6\textwidth]{angled}
	\caption{The ROV is hanging from the same point on the surface vessel with an equal tether. Note how the height of the ROV has changed because the tether has stayed the same}
	\label{fig:angled}
\end{figure}

Further, this non-perpendicular arrangement will lead to tangential forces, shown in \cref{fig:angled-force}. The forces will of course be equal and opposite on the vessel's end, though this has been omitted from the figure for clarity. The horizontal force the tether imparts on the ROV will act as a restoring force, trying to move it back to be perpendicular to the surface vessel. The surface vessel likewise will experience a pull towards the ROV. 

\begin{figure}
	\centering
	\includegraphics[width=0.6\textwidth]{angled-component}
	\caption{The component forces on the ROV resulting from an angled lift, decomposed}
	\label{fig:angled-force}
\end{figure}

Because these two vessels are connected, and therefore dependent on each other's positions, the control system needs to take this into account. Probably, the simplest solution will be to have one large control system handling both, or alternately having one of the vessels take a leading part and the other attempt to follow. This will be touched on later in this report, though not discussed in detail.

\section{Statement of intent}
For this project I want to create a simulation which is able to quantify the effects that the parts of the system have on each other. I want to be able to measure the tension in the wire, the force exerted by the vessels, how well the vessels follow each other or orders given and influence from the environment. The goal of this simulation is to be used as a future design tool for finalizing design of the Plan Sea vessel, its control systems, as well as defining operational criteria. 
\end{document}
