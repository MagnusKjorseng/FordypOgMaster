The stated goal of the project has been to make a simulation of the Plan Sea project's proposed implementation. The project consists of a small surface vessel and a non-buoyant remotely operated vehicle attached to each other by a lifting wire. The simulation has been made to allow for control system design, as well as to act as an engineering tool for this specific project, allowing different sizes and types of wires to be used, allowing for changing sea-state and currents, or the type and size of vessels. 

In this report I've documented why the simulation would be a helpful tool for prototyping. I've also described some of the reasons why I believe simulation is more helpful for this case than finding analytical solutions to the problems posed. 

I have created the simulation and run some validation tests on it to see whether it acts close to as should be expected from a realistic simulation. I have also created a simple PD-control system which is able to position the surface vessel at desired points in the world-space. The results found show that the simulation is likely more accurate than the simple analytical methods used to estimate the forces that would act on the system. 

I have proposed a list of topics for future iterations of the simulator, as well as stating my goal of continuing work on the simulator to use it for further development of the Plan Sea project. 

